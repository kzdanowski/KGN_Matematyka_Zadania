\documentclass[12pt]{article}
\usepackage{amssymb}
\usepackage{amsmath}
\usepackage{graphicx} % Required for inserting images
%\usepackage{fitch}
%\usepackage[english]{babel}
%\usepackage[polish, english]{babel}
\usepackage{polski}
%\usepackage[latin2]{inputenc}
\usepackage[utf8]{inputenc}
\usepackage{times}
\usepackage[T1]{fontenc}
\usepackage{listings}
\usepackage{datatool}
%\usepackage{tikz}
%\usetikzlibrary{decorations.markings}
%\usepackage{tikz-er2}
%\usepackage{fitch}


\usepackage{tikz}
\usepackage{pgflibraryshapes}
\usetikzlibrary{shapes}
\usetikzlibrary{arrows}
\usetikzlibrary{trees}
\usetikzlibrary{decorations.pathmorphing}
\usetikzlibrary{backgrounds}
\usetikzlibrary{positioning}
%\usetikzlibrary{placments}
\usetikzlibrary{fit}
\usepackage{algorithmic}
\usepackage{algorithm}


%%%% GLOBAL PARAMETERS %%%

\newcommand{\Rec}{\text{Rec}}
\newcommand{\Poly}{\text{Poly}}
\newcommand{\TIME}{\text{TIME}}
\newcommand{\poly}{\mathrm{poly}}
\newcommand{\set}[1]{{\{#1\}}}
\newcommand{\bin}{\text{bin}}
\newcommand{\st}{\text{st}}
\newcommand{\sqn}{\text{sq}}
\newcommand{\oneseq}[2]{{{#1}_{1},\dots, {#1}_{#2}}}


\let\gcd\undefined
\DeclareMathOperator{\gcd}{gcd}
\DeclareMathOperator{\lcm}{lcm}

\newcommand{\bit}{{\text{bit}}}
\newcommand{\isfunc}[3]{{{#1}\colon{#2}\rightarrow{#3}}}



\newcommand{\trans}[2]{{{F}^{#1}_{#2}}}
\newcommand{\floor}[1]{{\lfloor #1\rfloor}}

\newcommand{\cR}{\mathcal{R}}
\newcommand{\cC}{\mathcal{C}}
\newcommand{\cM}{\mathcal{M}}
\newcommand{\cG}{\mathcal{G}}
\newcommand{\cE}{\mathcal{E}}
\newcommand{\cH}{\mathcal{H}}
\newcommand{\cN}{\mathcal{N}}
\newcommand{\cF}{\mathcal{F}}
\newcommand{\cP}{\mathcal{P}}


\newcommand{\PA}{\text{PA}}

\newcommand{\bN}{\mathbb{N}}
\newcommand{\bbZ}{\mathbb{Z}}
\newcommand{\bbN}{\mathbb{N}}
\newcommand{\bR}{\mathbb{R}}

\DeclareMathOperator{\id}{id}
\newcommand{\Out}{{\mathit{Out}}}
\newcommand{\Move}{{\mathit{Move}}}
\newcommand{\update}{{\mathit{update}}}
\newcommand{\init}{{\mathit{init}}}
\newcommand{\choice}{\mathit{choice}}
\newcommand{\lex}{{\mathit{lex}}}
\newcommand{\lelex}{{<_{\lex}}}
\newcommand{\leqlex}{{\leq_{\lex}}}
\newcommand{\lesuf}{{<_{\mathrm{suf}}}}
\newcommand{\leqsuf}{{\leq_{\mathrm{suf}}}}
\newcommand{\Seq}{{\mathrm{Seq}}}
\newcommand{\Reach}{{\mathrm{Reach}}}
\newcommand{\New}{{\mathrm{New}}}
\newcommand{\Recurring}{{\mathrm{Recurring}}}
\newcommand{\Reconstruct}{{\mathrm{Reconstruct}}}

\newcommand{\child}{{\mathrm{child}}}
\newcommand{\children}{{\mathrm{children}}}

\newcommand{\uarrow}[1]{{\stackrel{#1}{\rightarrow}}}

\newcommand{\Buchi}{{B\"{u}chi\ }}

\newcommand{\move}[3]{{{#1}\uarrow{#2}{\set{#3}}}}
\newcommand{\moveset}[3]{{{#1}\uarrow{#2}{#3}}}
\newcommand{\movestop}[1]{{{#1}\uarrow{}{\bot}}}

\newcommand\sltit[1]{\begin{center}{ \bf #1}\end{center}}

\newcommand{\zf}[2]{{{#1}\rightarrow{#2}}}
\newcommand{\zw}[2]{{{#1}\twoheadrightarrow{#2}}}

%\newcommand{\set}[1]{{\{#1\}}}
\newcommand{\imp}{\Rightarrow}

%%%%%%%%%%%%%%%%%%%%%%%%%%%

\newcommand{\bluemath}[1]{{\setbeamercolor{math text}{fg=blue}{#1}}}
\newcommand{\greenmath}[1]{{\setbeamercolor{math text}{fg=green}{#1}}}
\newcommand{\redmath}[1]{{\setbeamercolor{math text}{fg=red}{#1}}}
\newcommand{\defintro}[1]{ {\color{red}{#1}}}
\newcommand{\tred}[1]{{\color{red}{#1}}}
\newcommand{\tblue}[1]{{\color{blue}{#1}}}
\newcommand{\tgreen}[1]{{\color{green}{#1}}}
\newcommand{\tyellow}[1]{{\color{yellow}{#1}}}
\newcommand{\intro}[1]{\emph{#1}}


\newcommand{\przyklad}{{\bf Przykład.\ }}


\newtheorem{thm}{Twierdzenie}
\newtheorem{crl}[thm]{Wniosek}
\newtheorem{lem}[thm]{Lemat}
\newtheorem{fact}[thm]{Fakt}
\newtheorem{prop}[thm]{Proposition}
\newtheorem{dfn}[thm]{Definicja}
\newtheorem{rem}[thm]{Remark}
\newtheorem{exmpl}[thm]{Example}
\newtheorem{exmpls}[thm]{Examples}
\newtheorem{post}[thm]{Postulate}

\newcommand{\ifff}{\Leftrightarrow}
\newcommand{\NN}{\mathbb{N}}
\newcommand{\RR}{\mathbb{R}}

\DeclareMathOperator{\card}{card}


\newcommand{\ft}[1]{\frametitle{#1}}
\newcommand{\lstl}[1]{\begin{lstlisting} #1\end{listing}}


%\lstdefinestyle{Oracle}{basicstyle=\ttfamily,
 %                       keywordstyle=\lstuppercase,
   %                     emphstyle=\itshape,
      %                  showstringspaces=false,
          %              }

%\lstdefinelanguage[Oracle]{SQL}[]{SQL}{
 %morekeywords={WITH, OVER, PARTITION, ACCESS, MOD, NLS_DATE_FORMAT, NVL, REPLACE, SYSDATE,
     %                    TRUNC, REFERENCES, DEFERRED,
         %           AFTER, REFERENCING, NEW, ROW, FOR, EACH, OLD, BEFORE, STATEMENT, OF, IF, return, is, cursor,
             %       BEGIN, END, LOOP, DECLARE, open, close, type, rowtype, fetch, exit, package, body, out, ref},
%}
%\lstdefi
%\lstset{language=[Oracle]SQL,  style=Oracle,  }


\title{Zadania \\ Matematyka I, Kognitywistyka,}
\author{Konrad Zdanowski}
%\date{13 luty 2016}
%\date{}

\begin{document}
\maketitle
\thispagestyle{empty}
%begin{center}
%{\bf \large ELT -- przykładowy egzamin 2017}\\
%\medskip
%Konrad Zdanowski
%\end{center}
%\vspace{5pt}

\section{Teoria liczb}
\subsection{Faktoryzacja, $\gcd$, $\lcm$}
\begin{enumerate}
    \item Czy 113, 201, 213 to liczby pierwsze? 
        (\cite[4.6.3, zad. 2]{Forman.2015})
     \item Znajdź faktoryzację: 3465, 40 320, 14641.  (\cite{Forman.2015})	
\item  Czy 1 111 111 111 jest pierwsza? (\cite{Forman.2015})
\item Niech $m = 2^2*3^3 * 5 * 7* 11$,  $n = 2*3*11$.
   Wyznacz $\gcd$, $\lcm$. (\cite{Forman.2015})
\item Niech $m = 5^2*7*11*13^2$, $n = 2*3*7^3*11^2*13$,
            $k=3*5*7^2*11^3$.
   Wyznacz $\gcd(m,n,k)$, $\lcm(m,n,k)$. 
\item 
 Czy jest nieskończenie wiele liczb pierwszych postaci $n^2 - 49$,
 dla pewnego $n\in\NN$? (\cite{Forman.2015})
\item 
Jeśli $p$ jest pierwsza, to czy $2^p - 1$ jest pierwsza? (\cite{Forman.2015})
\item  (\cite{Forman.2015})
 Wyznacz
    $\gcd(756, 2205),
    \gcd(4725, 17460), 
    \gcd(465, 3861),
    \gcd(4600, 2116), \\
    \gcd(630, 990), 
    \gcd(96, 144)$.

Wyznacz
$\lcm(756, 2205),
\lcm(4725, 17460),
\lcm(465, 3861),
\lcm(4600, 2116),\\
\lcm(630, 990),
\lcm(96, 144)$.


\item Wyznacz wszystkie liczby, których nie dzieli
żadna liczba pierwsza większa od pięciu i które mają 
dokładnie pięć dzielników.

\item Wyznacz wszystkie liczby, które dzielą $5*7$. 
      Ile jest takich liczb?


\end{enumerate}



\subsection{Przystawanie modulo}
\begin{enumerate}
    \item Rozstrzygnij, czy jest prawdą
    \begin{itemize}
        \item
    $0 \equiv 6 \pmod 3$,
    \item
 $35 \equiv 55 \pmod 9$,
\item  $(-23) \equiv 20 \pmod 7$
 \item $(-3) \equiv 3 \pmod 6$,
 \item $(-2) \equiv 2 \pmod 3$,
 \item $16 \equiv 185 \pmod 11.$
 \end{itemize}
    (\cite[sec. 6.1, p. 154]{Forman.2015})



\item
Wyznacz wszystkie liczby $n\in \bbZ$ takie, że
$n\equiv 2 \pmod 5$.
\item 
Czy $(-1)\equiv 1 \pmod 2$?

\item Wyznacz resztę z dzielenia liczby $17*23*45$ przez $8$.
 Wyznacz resztę z dzielenia liczby $17*23*45$ przez $5$.

Nie używaj kalkulatora.

\item Znajdź $n\in \NN$ takie, że $n\equiv 3 \pmod 5$ i 
$n\equiv 2 \pmod 3$.

\item Znajdź $n\in \NN$ takie, że $n\equiv 4 \pmod 4$ i 
$n\equiv 2 \pmod 5$.

\item Nie znajdź $n\in \NN$ takiego, że
$n\equiv 3 \pmod 6$ is $n\equiv 0 \pmod 2$.

Dlaczego takie $n$ nie istnieje?

\item Nie znajdź $n\in \NN$ takiego, że
$n\equiv 3 \pmod 6$ is $n\equiv 2 \pmod 9$.

Dlaczego takie $n$ nie istnieje?

\item Wyznacz wszystkie liczby, które dzielą $5*7$. 
 

\item Korzystając z Twierdzenia Eulera ($a^{\varphi(n)} \equiv 1 \pmod n$, gdy $\gcd(a,n)=1$) i Małego Twierdzenia Fermata 
($x^{p-1}\equiv 1 \pmod p$, dla liczby pierwszej $p$)
i z tego, że relacja przystawania modulo jest kongruencją
względem dodawania i mnożenia,
oblicz
\begin{itemize}
    \item $3^{100} \bmod 5$,
    \item $5^{100} \bmod 7$,
    \item $3^{100}\bmod 10$,
    \item $3^{100}\bmod 6$ (uwaga),
    \item $4^{100}\bmod 9$,
    \item $2^{2^{100}} \bmod 5$,
    \item $5^{5^{100}} \bmod 3$.
\end{itemize}

\end{enumerate}

\subsection{Indukcja}\label{ss:indukcja}
\begin{enumerate}
    \item Udowodnij $\Sigma_{i=1}^n i = \frac{n(n+1)}{2}$. 
    \item Udowodnij $\Sigma_{i=0}^n (2i+1) = (n+1)^2$
           (sumę $n$ pierwszych liczb nieparzystych).
    \item Udowodnij, dla każdego $n\geq 1$, dla wszystkich $x_1,\dots, x_n\in \bR$, $|x_1 + ... + x_n| \leq  |x_1| + ... + |x_n|$.
\item 
Dla dowolnego  $n\geq 1$, $\forall x \in (0,1)\, x^n \leq  x$.

Skorzystaj z faktu, że dla dowolnych $a,b$, jeśli $0\leq a < 1$ i $b\geq 0$, to $ab<b$.

\item 
Udowodnij, że dla dowolnego $n$, $\Sigma_{i=0}^n \frac{1}{2^i} \leq 2$.

Rozważ wzmocnienie  tezy, 
do $\Sigma_{i=0}^n \frac{1}{2^i} \leq 2-\frac{1}{2^{n}}$.

\item \label{ind:2nn2} Udowodnij, że dla dowolnego $n\geq 4$, $2^n\geq  n^2$.

Której części dowodu indukcyjnego nie można przeprowadzić 
dla tezy $\forall n\geq 0\, (2^n\geq n^2)$.

Której części dowodu indukcyjnego nie można przeprowadzić 
dla tezy $\forall n\geq 3\, (2^n\geq n^2)$.

\item Ciąg Fibbonacciego definiujmy jako 
$F(1)=F(2)=1$, oraz $F(n+2)=F(n)+F(n+1)$ dla $n\geq 1$.


Udowodnij, że dla $n\geq 1$, 
$$
F(n)=\frac{1}{\sqrt{5}}\left(\frac{1+\sqrt{5}}{2}\right)^n- 
    \frac{1}{\sqrt{5}}\left(\frac{1-\sqrt{5}}{2}\right)^n.
$$

\item 
(Nierówność  Bernoulliego,  uproszczony przypadek) Dla dowolnego $n\geq 1$, 
\[
\forall x \geq 0\, ((1+x)^n \geq (1+nx)).
\]

\end{enumerate}

\section{Uwagi lub (p)odpowiedzi}


\begin{itemize}
    \item Część~\ref{ss:indukcja}, zadanie~\ref{ind:2nn2}. 
     
     W tezie $\forall n\geq 0 (2^n\geq n^2)$ nie uda się 
     udowodnić kroku indukcyjnego.
     
     W tezie $\forall n\geq 3 (2^n\geq n^2)$ krok indukcyjny da się udowodnić, ale nie da się udowodnić przypadek bazowy.
\end{itemize}

\bibliographystyle{alpha}
\bibliography{Mat_KGN}

\end{document}	
