\documentclass[12pt]{article}
\usepackage{amssymb}
\usepackage{amsmath}
\usepackage{amsthm}
\usepackage{graphicx} % Required for inserting images
%\usepackage{fitch}
%\usepackage[english]{babel}
%\usepackage[polish, english]{babel}
\usepackage{polski}
%\usepackage[latin2]{inputenc}
\usepackage[utf8]{inputenc}
\usepackage{times}
\usepackage[T1]{fontenc}
\usepackage{listings}
\usepackage{datatool}
%\usepackage{tikz}
%\usetikzlibrary{decorations.markings}
%\usepackage{tikz-er2}
%\usepackage{fitch}


\usepackage{tikz}
\usepackage{pgflibraryshapes}
\usetikzlibrary{shapes}
\usetikzlibrary{arrows}
\usetikzlibrary{trees}
\usetikzlibrary{decorations.pathmorphing}
\usetikzlibrary{backgrounds}
\usetikzlibrary{positioning}
%\usetikzlibrary{placments}
\usetikzlibrary{fit}
\usepackage{algorithmic}
\usepackage{algorithm}


%%%% GLOBAL PARAMETERS %%%

\newcommand{\Rec}{\text{Rec}}
\newcommand{\Poly}{\text{Poly}}
\newcommand{\TIME}{\text{TIME}}
\newcommand{\poly}{\mathrm{poly}}
\newcommand{\set}[1]{{\{#1\}}}
\newcommand{\bin}{\text{bin}}
\newcommand{\st}{\text{st}}
\newcommand{\sqn}{\text{sq}}
\newcommand{\oneseq}[2]{{{#1}_{1},\dots, {#1}_{#2}}}


\let\gcd\undefined
\DeclareMathOperator{\gcd}{gcd}
\DeclareMathOperator{\lcm}{lcm}

\newcommand{\bit}{{\text{bit}}}
\newcommand{\isfunc}[3]{{{#1}\colon{#2}\rightarrow{#3}}}



\newcommand{\trans}[2]{{{F}^{#1}_{#2}}}
\newcommand{\floor}[1]{{\lfloor #1\rfloor}}

\newcommand{\cR}{\mathcal{R}}
\newcommand{\cC}{\mathcal{C}}
\newcommand{\cM}{\mathcal{M}}
\newcommand{\cG}{\mathcal{G}}
\newcommand{\cE}{\mathcal{E}}
\newcommand{\cH}{\mathcal{H}}
\newcommand{\cN}{\mathcal{N}}
\newcommand{\cF}{\mathcal{F}}
\newcommand{\cP}{\mathcal{P}}


\newcommand{\PA}{\text{PA}}

\newcommand{\bN}{\mathbb{N}}
\newcommand{\bbZ}{\mathbb{Z}}
\newcommand{\bbN}{\mathbb{N}}
\newcommand{\bR}{\mathbb{R}}

\DeclareMathOperator{\id}{id}
\newcommand{\Out}{{\mathit{Out}}}
\newcommand{\Move}{{\mathit{Move}}}
\newcommand{\update}{{\mathit{update}}}
\newcommand{\init}{{\mathit{init}}}
\newcommand{\choice}{\mathit{choice}}
\newcommand{\lex}{{\mathit{lex}}}
\newcommand{\lelex}{{<_{\lex}}}
\newcommand{\leqlex}{{\leq_{\lex}}}
\newcommand{\lesuf}{{<_{\mathrm{suf}}}}
\newcommand{\leqsuf}{{\leq_{\mathrm{suf}}}}
\newcommand{\Seq}{{\mathrm{Seq}}}
\newcommand{\Reach}{{\mathrm{Reach}}}
\newcommand{\New}{{\mathrm{New}}}
\newcommand{\Recurring}{{\mathrm{Recurring}}}
\newcommand{\Reconstruct}{{\mathrm{Reconstruct}}}

\newcommand{\child}{{\mathrm{child}}}
\newcommand{\children}{{\mathrm{children}}}

\newcommand{\uarrow}[1]{{\stackrel{#1}{\rightarrow}}}

\newcommand{\Buchi}{{B\"{u}chi\ }}

\newcommand{\move}[3]{{{#1}\uarrow{#2}{\set{#3}}}}
\newcommand{\moveset}[3]{{{#1}\uarrow{#2}{#3}}}
\newcommand{\movestop}[1]{{{#1}\uarrow{}{\bot}}}

\newcommand\sltit[1]{\begin{center}{ \bf #1}\end{center}}

\newcommand{\zf}[2]{{{#1}\rightarrow{#2}}}
\newcommand{\zw}[2]{{{#1}\twoheadrightarrow{#2}}}

%\newcommand{\set}[1]{{\{#1\}}}
\newcommand{\imp}{\Rightarrow}

%%%%%%%%%%%%%%%%%%%%%%%%%%%

\newcommand{\bluemath}[1]{{\setbeamercolor{math text}{fg=blue}{#1}}}
\newcommand{\greenmath}[1]{{\setbeamercolor{math text}{fg=green}{#1}}}
\newcommand{\redmath}[1]{{\setbeamercolor{math text}{fg=red}{#1}}}
\newcommand{\defintro}[1]{ {\color{red}{#1}}}
\newcommand{\tred}[1]{{\color{red}{#1}}}
\newcommand{\tblue}[1]{{\color{blue}{#1}}}
\newcommand{\tgreen}[1]{{\color{green}{#1}}}
\newcommand{\tyellow}[1]{{\color{yellow}{#1}}}
\newcommand{\intro}[1]{\emph{#1}}


\newcommand{\przyklad}{{\bf Przykład.\ }}


\newtheorem{thm}{Twierdzenie}
\newtheorem{crl}[thm]{Wniosek}
\newtheorem{lem}[thm]{Lemat}
\newtheorem{fact}[thm]{Fakt}
\newtheorem{prop}[thm]{Stwierdzenie}
\newtheorem{dfn}[thm]{Definicja}
\newtheorem{rem}[thm]{Remark}
\newtheorem{exmpl}[thm]{Example}
\newtheorem{exmpls}[thm]{Examples}
\newtheorem{post}[thm]{Postulate}

\newcommand{\ifff}{\Leftrightarrow}
\newcommand{\NN}{\mathbb{N}}
\newcommand{\RR}{\mathbb{R}}

\DeclareMathOperator{\card}{card}
\DeclareMathOperator{\dom}{dom}
\DeclareMathOperator{\rng}{rng}


\newcommand{\ft}[1]{\frametitle{#1}}
\newcommand{\lstl}[1]{\begin{lstlisting} #1\end{listing}}

\newcommand{\ala}{\textrm{ala}}
\newcommand{\ela}{\textrm{ela}}
\newcommand{\ola}{\textrm{ola}}


%\lstdefinestyle{Oracle}{basicstyle=\ttfamily,
 %                       keywordstyle=\lstuppercase,
   %                     emphstyle=\itshape,
      %                  showstringspaces=false,
          %              }

%\lstdefinelanguage[Oracle]{SQL}[]{SQL}{
 %morekeywords={WITH, OVER, PARTITION, ACCESS, MOD, NLS_DATE_FORMAT, NVL, REPLACE, SYSDATE,
     %                    TRUNC, REFERENCES, DEFERRED,
         %           AFTER, REFERENCING, NEW, ROW, FOR, EACH, OLD, BEFORE, STATEMENT, OF, IF, return, is, cursor,
             %       BEGIN, END, LOOP, DECLARE, open, close, type, rowtype, fetch, exit, package, body, out, ref},
%}
%\lstdefi
%\lstset{language=[Oracle]SQL,  style=Oracle,  }


\title{Matematyka I, Kognitywistyka,\\
        Zadania}
\author{Konrad Zdanowski}
%\date{13 luty 2016}
%\date{}

\begin{document}
\maketitle
\thispagestyle{empty}
%begin{center}
%{\bf \large ELT -- przykładowy egzamin 2017}\\
%\medskip
%Konrad Zdanowski
%\end{center}
%\vspace{5pt}

Tekst ten zawiera listę najbardziej istotnych definicji
i twierdzeń z wykładu oraz przykładowe zadania.

\section{Teoria liczb}
\subsection{Faktoryzacja, $\gcd$, $\lcm$}
\begin{enumerate}
    \item Czy 113, 201, 213 to liczby pierwsze? 
        (\cite[4.6.3, zad. 2]{Forman.2015})
     \item Znajdź faktoryzację: 3465, 40 320, 14641.  (\cite{Forman.2015})	
\item  Czy 1 111 111 111 jest pierwsza? (\cite{Forman.2015})
\item Niech $m = 2^2*3^3 * 5 * 7* 11$,  $n = 2*3*11$.
   Wyznacz $\gcd$, $\lcm$. (\cite{Forman.2015})
\item Niech $m = 5^2*7*11*13^2$, $n = 2*3*7^3*11^2*13$,
            $k=3*5*7^2*11^3$.
   Wyznacz $\gcd(m,n,k)$, $\lcm(m,n,k)$. 
\item 
 Czy jest nieskończenie wiele liczb pierwszych postaci $n^2 - 49$,
 dla pewnego $n\in\NN$? (\cite{Forman.2015})
\item 
Jeśli $p$ jest pierwsza, to czy $2^p - 1$ jest pierwsza? (\cite{Forman.2015})
\item  (\cite{Forman.2015})
 Wyznacz
    $\gcd(756, 2205),
    \gcd(4725, 17460), 
    \gcd(465, 3861),
    \gcd(4600, 2116), \\
    \gcd(630, 990), 
    \gcd(96, 144)$.

Wyznacz
$\lcm(756, 2205),
\lcm(4725, 17460),
\lcm(465, 3861),
\lcm(4600, 2116),\\
\lcm(630, 990),
\lcm(96, 144)$.


\item Wyznacz wszystkie liczby, których nie dzieli
żadna liczba pierwsza większa od pięciu i które mają 
dokładnie pięć dzielników.

\item Wyznacz wszystkie liczby, które dzielą $5*7$. 
      Ile jest takich liczb?


\end{enumerate}



\subsection{Przystawanie modulo}\label{subsec:modulo}
\begin{enumerate}
    \item Rozstrzygnij, czy jest prawdą
    \begin{itemize}
        \item
    $0 \equiv 6 \pmod 3$,
    \item
 $35 \equiv 55 \pmod 9$,
\item  $(-23) \equiv 20 \pmod 7$
 \item $(-3) \equiv 3 \pmod 6$,
 \item $(-2) \equiv 2 \pmod 3$,
 \item $16 \equiv 185 \pmod 11.$
 \end{itemize}
    (\cite[sec. 6.1, p. 154]{Forman.2015})



\item
Wyznacz wszystkie liczby $n\in \bbZ$ takie, że
$n\equiv 2 \pmod 5$.
\item 
Czy $(-1)\equiv 1 \pmod 2$?

\item Wyznacz resztę z dzielenia liczby $17*23*45$ przez $8$.
 Wyznacz resztę z dzielenia liczby $17*23*45$ przez $5$.

Nie używaj kalkulatora.

\item Znajdź $n\in \NN$ takie, że $n\equiv 3 \pmod 5$ i 
$n\equiv 2 \pmod 3$.

\item Znajdź $n\in \NN$ takie, że $n\equiv 4 \pmod 4$ i 
$n\equiv 2 \pmod 5$.

\item Nie znajdź $n\in \NN$ takiego, że
$n\equiv 3 \pmod 6$ is $n\equiv 0 \pmod 2$.

Dlaczego takie $n$ nie istnieje?

\item Nie znajdź $n\in \NN$ takiego, że
$n\equiv 3 \pmod 6$ is $n\equiv 2 \pmod 9$.

Dlaczego takie $n$ nie istnieje?

\item Wyznacz wszystkie liczby, które dzielą $5*7$. 
 

\item \label{exc:modpowers}

 Korzystając z Twierdzenia Eulera ($a^{\varphi(n)} \equiv 1 \pmod n$, gdy $\gcd(a,n)=1$) i Małego Twierdzenia Fermata 
($x^{p-1}\equiv 1 \pmod p$, dla liczby pierwszej $p$)
i z tego, że relacja przystawania modulo jest kongruencją
względem dodawania i mnożenia,
oblicz
\begin{itemize}
    \item $3^{100} \bmod 5$,
    \item $5^{100} \bmod 7$,
    \item $3^{100}\bmod 10$,
    \item $3^{100}\bmod 6$ (uwaga),
    \item $4^{100}\bmod 9$,
    \item $2^{2^{100}} \bmod 5$,
    \item $5^{5^{100}} \bmod 3$.
\end{itemize}

\end{enumerate}

\subsection{Indukcja}\label{ss:indukcja}
\begin{enumerate}
    \item Udowodnij $\Sigma_{i=1}^n i = \frac{n(n+1)}{2}$. 
    \item Udowodnij $\Sigma_{i=0}^n (2i+1) = (n+1)^2$
           (sumę $n$ pierwszych liczb nieparzystych).
    \item Udowodnij, dla każdego $n\geq 1$, dla wszystkich $x_1,\dots, x_n\in \bR$, $|x_1 + ... + x_n| \leq  |x_1| + ... + |x_n|$.
\item 
Dla dowolnego  $n\geq 1$, $\forall x \in (0,1)\, x^n \leq  x$.

Skorzystaj z faktu, że dla dowolnych $a,b$, jeśli $0\leq a < 1$ i $b\geq 0$, to $ab<b$.

\item 
Udowodnij, że dla dowolnego $n$, $\Sigma_{i=0}^n \frac{1}{2^i} \leq 2$.

Rozważ wzmocnienie  tezy, 
do $\Sigma_{i=0}^n \frac{1}{2^i} \leq 2-\frac{1}{2^{n}}$.

\item \label{ind:2nn2} Udowodnij, że dla dowolnego $n\geq 4$, $2^n\geq  n^2$.

Której części dowodu indukcyjnego nie można przeprowadzić 
dla tezy $\forall n\geq 0\, (2^n\geq n^2)$.

Której części dowodu indukcyjnego nie można przeprowadzić 
dla tezy $\forall n\geq 3\, (2^n\geq n^2)$.

\item Ciąg Fibbonacciego definiujmy jako 
$F(1)=F(2)=1$, oraz $F(n+2)=F(n)+F(n+1)$ dla $n\geq 1$.


Udowodnij, że dla $n\geq 1$, 
$$
F(n)=\frac{1}{\sqrt{5}}\left(\frac{1+\sqrt{5}}{2}\right)^n- 
    \frac{1}{\sqrt{5}}\left(\frac{1-\sqrt{5}}{2}\right)^n.
$$

\item 
(Nierówność  Bernoulliego,  uproszczony przypadek) Dla dowolnego $n\geq 1$, 
\[
\forall x \geq 0\, ((1+x)^n \geq (1+nx)).
\]

\end{enumerate}

\section{Teoria relacji i funkcji}

\begin{dfn}
	Niech $X$, $Y$ będą dowolnymi zbiorami. Iloczyn kartezjański $X\times Y$ 
	to zbiór par uporządkowanych postaci $(a,b)$ gdzie $a\in X$ i $b\in Y$.
	
	Relacja między zbiorami $X$ i $Y$ to dowolny podzbiór $R$
	iloczynu kartezjańskiego $X\times Y$.
	
	Jeśli $R\subseteq X\times X$ to mówimy, że $R$ jest relacją na $X$.
\end{dfn}


Możemy wyróżnić relację pełną, równą $X\times Y$, relację pustą $\emptyset$.
Wyróżniamy też relację identycznościową na $X$, 
$$
\id_X= \{(a,a)\colon a\in X\}.
$$ 

\begin{dfn}
	Dziedzina relacji $\dom(R)$.
	
	Przeciwdziedzina (obraz) relacji $\rng(R)$.
\end{dfn}

\subsection{Jak możemy przedstawić relację?}

\begin{enumerate}
	\item Zbiór par,
	\item rysunek,
	\item macierz.
\end{enumerate}

Następujące własności relacji bedą dla nas istotne.

\begin{dfn}
Niech $R$ będzie relacją na $X$. $R$ jest 
\begin{itemize}
	\item zwrotna,
	\item przeciwzwrotna,
	\item symetryczna,
	\item przeciwsymetryczna,
	\item antysymetryczna,
	\item przechodnia.
\end{itemize}	
\end{dfn}

\begin{dfn}[Rodzaje relacji]
	\begin{itemize}
		\item Relacja równoważności,
		\item Relacja (częściowego) porządku
		\item Relacja liniowego porządku
	\end{itemize}
\end{dfn}


\begin{dfn}[Klasa abstrakcji]
\end{dfn}

\subsection{Operacje na relacjach}

\begin{dfn}
	\begin{itemize}
		\item relacja odwrotna $R^{-1}$,
		\item złożenie relacji $R\circ S$,
		\item tranzytywne domknięcie $R^*$.
	\end{itemize}
\end{dfn}

	
\subsection{Funkcje}


\subsection{Zadania (ChatGPT)}

\begin{enumerate} 

\item 
Relacja na zbiorze liczb całkowitych.

Rozważ zbiór liczb całkowitych $\mathbb{Z}$. Zdefiniuj relację $R$ na $\mathbb{Z}$ jako: 
\[
x \, R \, y \text{ wtedy i tylko wtedy, gdy } x = y.
\]
Udowodnij, że relacja $R$ jest zwrotna, symetryczna i przechodnia.

\item 
 Relacja "większe niż"

Rozważ zbiór liczb naturalnych $ \mathbb{N} $. Zdefiniuj relację $ R $ na $ \mathbb{N} $ jako:
\[
x \, R \, y \text{ wtedy i tylko wtedy, gdy } x > y.
\]
Sprawdź, czy relacja $ R $ jest antysymetryczna i przechodnia.
Udowodnij, że ta relacja nie jest zwrotna.

\item
 Relacja na zbiorze osób.

Rozważ zbiór osób $ P = \{A, B, C\} $. Zdefiniuj relację $ R $ na $ P $ jako:
\[
x \, R \, y \text{ wtedy i tylko wtedy, gdy osoba } x \text{ jest starsza od osoby } y.
\]
Sprawdź, czy relacja $ R $ jest antysymetryczna i przechodnia.
Czy relacja jest zwrotna? Uzasadnij odpowiedź.

\item
 Relacja podzielności.

Rozważ zbiór liczb naturalnych $ \mathbb{N} $. Zdefiniuj relację $ R $ na $ \mathbb{N} $ jako:
\[
x \, R \, y \text{ wtedy i tylko wtedy, gdy } x \text{ jest podzielne przez } y.
\]
Sprawdź, czy relacja $ R $ jest zwrotna, symetryczna i przechodnia.

\item 
 Relacja ,,przyjaciele''.

Rozważ zbiór osób $ P $ i zdefiniuj relację $ R $ na $ P $ jako:
\[
x \, R \, y \text{ wtedy i tylko wtedy, gdy osoby } x \text{ i } y \text{ są przyjaciółmi}.
\]
 Czy relacja $ R $ jest symetryczna? Uzasadnij odpowiedź.
Czy relacja $ R $ jest zwrotna? Uzasadnij odpowiedź.

\item 
Relacja ,,parzystość'',

Rozważ zbiór liczb naturalnych $ \mathbb{N} $. Zdefiniuj relację $ R $ na $ \mathbb{N} $ jako:
\[
x \, R \, y \text{ wtedy i tylko wtedy, gdy suma liczb } x \text{ i } y \text{ jest liczbą parzystą.}
\]
Sprawdź, czy relacja $ R $ jest zwrotna, symetryczna i przechodnia.

\item 
 Relacja równości.

Rozważ zbiór $ S = \{1, 2, 3\} $. Zdefiniuj relację $ R $ na $ S $ jako:
\[
x \, R \, y \text{ wtedy i tylko wtedy, gdy } x = y.
\]
Sprawdź, czy relacja $ R $ jest zwrotna, symetryczna, przechodnia i antysymetryczna.

\item 
Relacja ,,dzielenie się przestrzenią''.

Rozważ zbiór punktów na płaszczyźnie $ \mathbb{R}^2 $. Zdefiniuj relację $ R $ na $ \mathbb{R}^2 $ jako:
\[
(x_1, y_1) \, R \, (x_2, y_2) \text{ wtedy i tylko wtedy, gdy } x_1 = x_2 \text{ i } y_1 = y_2.
\]
Sprawdź, czy relacja $ R $ jest zwrotna, symetryczna, przechodnia i antysymetryczna.

\item 
Relacja ,,bycie większym lub równym''.

Rozważ zbiór liczb całkowitych $ \mathbb{Z} $. Zdefiniuj relację $ R $ na $ \mathbb{Z} $ jako:
\[
x \, R \, y \text{ wtedy i tylko wtedy, gdy } x \geq y.
\]
Sprawdź, czy relacja $ R $ jest zwrotna, symetryczna, przechodnia i antysymetryczna.

\item 
 Relacja na zbiorze liter.

Rozważ zbiór liter $ A, B, C, D $. Zdefiniuj relację $ R $ jako:
\[
x \, R \, y \text{ wtedy i tylko wtedy, gdy litera } x \text{ sasiąduje z } y \text{ w alfabecie.}
\]
Sprawdź, czy relacja $ R $ jest zwrotna, symetryczna, antysymetryczna i przechodnia.

\item 
Własności relacji na zbiorze liczb  

Niech $ R $ będzie relacją na zbiorze $ A = \{1, 2, 3, 4\} $ zdefiniowaną jako $ a R b $, jeśli $ a \leq b $.  
Ustal, czy relacja $ R $ jest zwrotna, symetryczna, przechodnia.

\item 
Relacja podzielności  

Niech $ R $ będzie relacją na zbiorze liczb całkowitych $ A = \{2, 3, 6, 9, 12\} $, gdzie $ a R b $ zachodzi, gdy $ a $ dzieli $ b $ bez reszty.  
Ustal, czy relacja $ R $ jest zwrotna, symetryczna, przechodnia.

\item 
Relacja równości modulo  

Rozważ zbiór $ A = \{0, 1, 2, 3, 4, 5\} $ oraz relację $ R $ zdefiniowaną jako $ a R b $, gdy $ a \equiv b \pmod{2} $.  
Ustal, czy relacja $ R $ jest zwrotna, symetryczna, przechodnia.
Jeśli tak, to jak wyglądają jej klasy abstrakcji.
\item 
Relacja na zbiorze punktów  

Na płaszczyźnie rozważmy zbiór punktów $ A = \{(0,0), (1,1), (2,2), (1,0), (2,1)\} $. Zdefiniuj relację $ R $ jako $ (x_1, y_1) R (x_2, y_2) $, jeśli $ x_1 = x_2 $.  
Ustal, czy relacja $ R $ jest zwrotna, symetryczna, przechodnia.

\item 
Relacja porządku  

Niech $ R $ będzie relacją na zbiorze $ A = \{a, b, c\} $ zdefiniowaną jako $ a R b $, jeśli $ a $ jest "mniejsze lub równe" od $ b $ według pewnej kolejności alfabetycznej.  
Sprawdź, czy relacja $ R $ jest relacją porządku częściowego.  
Zapisz diagram Hassego dla tej relacji.

\item 
 Właściwości relacji na zbiorze liczb  

Rozważ zbiór $ A = \{1, 2, 3, 4, 5\} $ oraz relację $ R $ zdefiniowaną jako $ a R b $ wtedy i tylko wtedy, gdy $ a + b $ jest liczbą parzystą.  
Sprawdź, czy relacja $ R $ jest zwrotna, symetryczna, przechodnia.  

\item 
Klasy abstrakcji  

Na zbiorze $ A = \{1, 2, 3, 4, 5, 6\} $ zdefiniowana jest relacja $ R $, gdzie $ a R b $, jeśli $ a - b $ jest podzielne przez 3.  
Wykaż, że $ R $ jest relacją równoważności.  
Wyznacz klasy abstrakcji dla tej relacji.

\item  Zapis relacji w postaci macierzy  

Rozważ zbiór $ A = \{1, 2, 3\} $ oraz relację $ R $ zdefiniowaną jako $ a R b $, gdy $ a + b $ jest liczbą nieparzystą.  
Przedstaw relację $ R $ w postaci macierzy.  
Na podstawie macierzy ustal, czy relacja $ R $ jest symetryczna.

\end{enumerate}

\section{Matematyka dyskretna}
\subsection{Zliczania}

Notacje. $|X|$ to moc zbioru $X$.
$\cP(X)$ to zbiór podzbiorów $X$.
$\cP^{=k}(X)$ to ilość $k$ elementowych podzbiorów 
zbioru $X$, gdzie $k\in\bN$.




\subsubsection{Zliczania zbiorów}



\begin{thm}
	Niech $X,Y, Z$ zbiory skończone. Wtedy 
	%\begin{enumerate} 
		%\item 
		$|X\cup Y|=|X|+|Y|-|X\cap Y|$ oraz
		%\item 
		 \begin{multline*}
		 	X\cup Y\cup Z|= |X|+|Y|+|Z|+ \\
		 	- |X\cap Y| - |X\cap Z| - |Y\cap Z| + |X\cap Y\cap Z|.
		 \end{multline*} 
		%\end{enumerate} 
\end{thm}

\begin{thm}
	Niech $X$ będzie zbiorem skończonym o mocy (liczności) $n$. 
	Wtedy $|\cP(X)|=2^n$.
\end{thm}

\begin{dfn}
	Dwumian Newtona to wyrażenie $\binom{n}{k}$, gdzie $0\leq k\leq n$, zdefiniowane jako
	\[
	\binom{n}{k}=\frac{n!}{k!(n-k)!}.
	\]
\end{dfn}
Ponieważ $0!=1$, to $\binom{0}{0}=1$.

\begin{thm}
	Dla $0\leq k \leq n$, ilość $k$-elementowych podzbiorów
	$n$ elementowego to~$\binom{n}{k}$. 
	
	Innymi słowy, jeśli $|X|=n$, 
	to $|\cP\sp{=k}(X)|=\binom{n}{k}$.
\end{thm}

Dwumian Newtona spełnia rekurencyjną zależność,
dla $k+1\leq n$, 
\[
\binom{n+1}{k+1} = \binom{n}{k+1} + \binom{n}{k}.  
\]

\subsubsection{Zliczania wyborów}


\begin{dfn}
	Niech $X$ będzie $n$ elementowym zbiorem.
	Wtedy $r$-kombinacja  zbioru $X$ to $r$ elementowy podzbiór $X$.
\end{dfn} 

Np. Jeśli $X$ jest zbiorem trzech osób,
$X=\{\ala, \ola, \ela \}$, to $2$-kombinacja
to dowolny dwuelementowy podzbiór $X$, 
np. $\{\ala, \ola \}$, $\{\ala, \ela\}$.

\begin{thm}
	Niech $X$ będzie $n$-elementowym zbiorem, niech $0\leq r\leq n$. Ilość $r$-kombinacji $X$ to $\binom{n}{r}$.
\end{thm}



Przykładowe $2$-kombinacje zbioru $\{\ala, \ela, \ola\}$,
to $\{\ala, \ela\}$, $\{\ala, \ola\}$.
(Uwaga, w zbiorach nie ma znaczenia kolejność wypisywania
elementów.)


\begin{dfn}
Niech $X$ będzie $n$ elementowym zbiorem .
Permutacja zbioru $X$ to sposób uporządkowania 
elementów $X$.
\end{dfn} 

Np. jeśli $X$ jest zbiorem $3$ osób, to możemy
na sześć sposobów ustawić te osoby w kolejce.



\begin{thm}
	Ilość permutacji zbioru $n$ elementowego,
	to $n!$.
\end{thm}


\begin{dfn}
	Niech $X$ będzie zbiorem $n$ elementowym i niech $r\leq n$.
	Wtedy, $r$-wariacja bez powtórzeń zbioru $X$ to sposób na wybranie i uporządkowanie $r$ różnych elementów z $X$.
\end{dfn} 

Przykładowe $2$-wariacje zbioru $\{\ala, \ela, \ola\}$,
to $(\ala, \ola)$, $(\ola, \ala)$, $(\ela, \ala)$.

\begin{thm}
	Ilość $r$-wariacji bez powtórzeń $n$ elementowego zbioru 
	to $\frac{n!}{(n-r)!}$.
\end{thm}

\begin{dfn}
	Niech $X$ będzie zbiorem $n$ elementowym i niech $r\leq n$.
	Wtedy, $r$-wariacja z powtórzeniami zbioru $X$ to sposób na wybranie i uporządkowanie $r$ elementóW z $X$, gdy elementy mogą się powtarzać.
\end{dfn} 

Przykładowe $2$-wariacje z powtórzeniami 
zbioru $\{\ala, \ela, \ola\}$,
to $(\ala, \ala)$, $(\ola, \ala)$, 
$(\ala, \ola)$, $(\ola, \ola)$.


\begin{thm}
	Ilość $r$-wariacji z powtórzeniami $n$ elementowego zbioru 
	to $n^r$.
\end{thm}

Jeśli losujemy kule tak jak w totolotku, to jest to kombinacja 
(kolejność wylosowania nie ma znaczenia).
Jeśli losujemy $r$ ponumerowanych kul 
i układamy je w rządek (bez zwracania do worka), to mamy
wariację bez powtórzeń. Jeśli zapisujemy wyniki kolejnych losowań a same kule wrzucamy z powrotem do worka, to mamy
wariację z powtórzeniami.
\bigskip


Dodatkowo, jeśli umieszczamy $n$ takich samych przedmiotów
w $r$ różnych pudełkach, to możemy zrobić to
na~$\binom{n+r-1}{r-1}$ sposobów.

Na przykład, jeśli mamy $1$, $2$, $5$ i $10$ groszówki, 
to możemy wybrać z nich $10$ moment 
na~$\binom{10+4-1}{4-1}=\binom{13}{3}$
sposobów.

\begin{thm}[Zasada szufladkowa]
	Niech $n\geq m\geq 1$. Jeśli $n$ przedmiotów umieścimy w 
	$m$ pudełkach, to będzie pudełko, w którym znajdzie się
	przynajmniej $\lceil \frac{n}{m}\rceil$ przedmiotów.
\end{thm}

\begin{crl}
	Jeśli $n+1$ przedmiotów umieścimy w $n$ pudełkach, gdzie $n\geq 1$, 
	to w pewnym pudełku znajdą się przynajmniej dwa przedmioty.
\end{crl}

\subsubsection{Zadania}
\begin{enumerate}
	\item 
	Zad. (\cite{MD}, Cw. 5.3.1, p.302)
	Wśród 200 osób 150 uprawia pływanie lub jogging lub oba sporty.
	85 uprawia pływanie, 60 uprawia pływanie i jogging. 
	Ile osób pływa?
	
	Czy informacja o ilości wszystkich osób była istotna?
	
	\item 
	Zad. 
	Ile liczb z $\{10, ..., 99\}$ ma dokładnie jedną liczbę równą $7$?
	Ile ma przynajmniej jedną siódemkę?
	Ile ma przynajmniej jedną $7$ lub $3$?
	Ile ma przynajmniej jedną $7$ i przynajmniej jedną $3$?
	
	\item 
	Ile liczb ze zbioru $\{1,...,100\}$ jest podzielnych przez $3$ i przez $5$?
	Ile przez $6$ i przez $9$?
	
	\item 
	 Na ile sposobów można usadzić $n$ osób na ławce?
	 \item 
	Na ile sposobów można usadzić $n$ osób przy okrągłym stole?
	\item  Na ile sposobów można rozdać $52$ karty po równo między $4$ graczy?
	\item Ile przekątnych ma $n$-kąt wypukły?
    \item Ile jest różnych sposobów ustawienia na półce dzieła $5$-tomowego tak, aby:
    \begin{itemize} 
	\item[a] tomy I i II stały obok siebie
	\item[b] tomy I i II nie stały obok siebie?
	\end{itemize} 
	\item Ile czteroosobowych komisji można stworzyć z grupy $9$ urzędników, jeżeli
	wiadomo, że wśród nich są osoby $A$ oraz $B$, które nie chcą razem pracować?
	\item Ile jest liczb całkowitych pomiędzy $1000$ a $9999$, których suma cyfr wynosi
	dokładnie $9$?
	\item  Na ile sposobów można podzielić $3n$ osób na $n$ grup $3$-osobowych?
	\item W klasie jest $n$ chłopców i $n$ dziewczynek. Na ile sposobów mogą utworzyć pary do tańca. 
	\item Jaka jest szansa trafienia szóstki w totolotku (losujemy sześć liczb z 49)? Ile powinna wynosić kumulacja, 
	 żeby przy cenie zakładu 3zł. opłacało się zagrać?
    \item Na ile sposobów można przejść z lewego górnego do prawego dolnego pola szachownicy, jeśli możemy poruszać się
    tylko w prawo i w dół?
    \item Na ile sposobów można odwiedzić wszystkie wierzchołki
    w grafie pełnym o $n$ wierzchołkach? 
    Na ile sposobów można to zrobić, ale tak, żeby wrócić do punktu wyjścia?
    \item Ile jest możliwych wyników rzutu dwiema rozróżnialnymi 
    kostkami do gry?
    
    Ile jest takich wyników, jeśli nie rozróżniamy kostek?
    
    Jaka jest szansa na wyrzucenie w sumie $12$ oczek? Jaka jest 
    szansa na wyrzucenie dwóch szóstek?
    Jaka jest szansa na wyrzucenie w sumie $11$?
    
    Jak jest szansa na wyrzucenie w sumie~$7$?
    Jak jest szansa na wyrzucenie w sumie~$8$?
    
    \item 
    Układamy ośmio literowe hasło z 26 liter alfabetu. Ile jest haseł, jeśli symbole mogą 
    się powtarzać. Ile jest takich haseł, w których musi wystąpić przynajmniej jedna 
    samogłoska (a, e, o, u, i). Ile jest haseł, w których musi wystąpić przynajmniej 
    jedna samogłoska i przynajmniej jedna spółgłoska.
    
    \item 
    W worku jest 20 kul ponumerowanych od $1$ do $20$. Losujemy (bez zwracania) pięć kul.
    Jaka jest szansa, że wylosujemy tylko kule o numerach parzystych?
    Jaka jest szansa, że wylosujemy przynajmniej jedną kulę o numerze parzystym?
    Jaka jest szansa, że wylosujemy przynajmniej jedną kulę o numerze parzystym
    i przynajmniej jedną kulę o numerze nieparzystym.

   Jak zmieni sie odpowiedź, jeśli będziemy zwracali po wylosowaniu kule do worka?
   

\end{enumerate}


\subsection{Zadania z ChatGPT}

\subsubsection{Zasada szufladkowa}

\begin{enumerate} 	\item %1. Uczniowie w klasie
	W klasie jest 26 uczniów, którzy przynieśli na zajęcia 10 różnych książek. Udowodnij, że co najmniej trzech uczniów przyniosło tę samą książkę.
	
	\item
	%2. Urodziny w grupie
	W grupie liczącej 13 osób udowodnij, że przynajmniej dwie osoby obchodzą urodziny w tym samym miesiącu.
	
	\item
	%3. Liczby w zbiorze
	Mamy dany zbiór 10 liczb naturalnych z zakresu od 1 do 18. Udowodnij, że istnieją dwie liczby w zbiorze, których różnica wynosi co najwyżej 2.
	
	\item
	%4. Monety w kieszeniach
	Pięciu przyjaciół ma razem 25 monet. Udowodnij, że przynajmniej jedna osoba ma co najmniej 5 monet.
	
	\item
	%5. Punkty na okręgu
	Na okręgu rozmieszczono 9 punktów. Udowodnij, że istnieją dwa punkty, których odległość (mierząc po łuku) nie jest większa niż 40 stopni.
	
	\item
	%6. Podział jabłek
	30 jabłek rozdzielono pomiędzy 8 koszyków. Udowodnij, że przynajmniej w jednym koszyku znajduje się co najmniej 4 jabłka.
	
	\item
	%7. Liczby w zbiorze liczb parzystych
	Wybieramy 8 różnych liczb całkowitych spośród liczb 2, 4, 6, …, 20. Udowodnij, że wśród wybranych liczb istnieją dwie, których różnica wynosi 4.
	
	\item
	%8. Kolory długopisów
	Na biurku znajduje się 12 długopisów w trzech różnych kolorach. Udowodnij, że przynajmniej cztery długopisy mają ten sam kolor.
	
	\item
	%9. Punkty w prostokącie
	W prostokącie umieszczono 6 punktów. Udowodnij, że można wybrać dwa punkty, których odległość jest mniejsza lub równa połowie długości przekątnej prostokąta.
	
	\item
	%10. Klocki o różnych rozmiarach
	Mamy 7 klocków, każdy o masie całkowitej (w kilogramach) z zakresu od 1 do 12 kg. Udowodnij, że można wybrać dwa klocki, których masy różnią się o co najwyżej 2 kg.
	
\end{enumerate}

\subsubsection{Permutacje}

\begin{enumerate}
	\item 
	%1. Permutacje liter słowa
	Ile różnych słów można utworzyć, przestawiając litery w słowie "KOT"?
	
	\item 
	%2. Permutacje z powtórzeniami
	Ile różnych słów można utworzyć, przestawiając litery w słowie "MAMA"?
	
	\item 
	%3. Siedzenia w rzędzie
	Czworo dzieci siada w jednym rzędzie na 4 miejscach. Na ile sposobów mogą się usiąść?
	
	\item 
	%4. Miejsca przy okrągłym stole
	Na ile różnych sposobów pięć osób może usiąść przy okrągłym stole?
\end{enumerate}

\subsubsection{Kombinacje}
\begin{enumerate}
	\item 
	%5. Wybór drużyny
	Z grupy 10 osób wybieramy drużynę składającą się z 4 osób. Na ile sposobów można wybrać drużynę?
	\item 
	%6. Wybór książek
	Na półce znajduje się 8 różnych książek. Ile sposobów jest na wybranie 3 książek, które zamierzasz przeczytać?
	\item 
	%7. Podzielenie obowiązków
	Sześcioro przyjaciół chce się podzielić na trzy dwuosobowe zespoły. Na ile sposobów można utworzyć takie zespoły?
\end{enumerate}

\subsubsection{Wariacje bez powtórzeń i z powtórzeniami}

\begin{enumerate}
	\item 
	%8. Wybór uczniów na podium
	Z grupy 5 uczniów wybieramy trzech, którzy zajmą miejsca na podium (złoto, srebro, brąz). Na ile sposobów można to zrobić?
	\item 
	%9. Kod dostępu
	Do zamka szyfrowego wybieramy 3-cyfrowy kod, gdzie cyfry nie mogą się powtarzać, a do wyboru mamy cyfry od 1 do 5. Na ile sposobów można ustawić taki kod?
	\item 
	%10. Układanie książek na półce
	Na półce mamy 7 różnych książek. Na ile sposobów możemy wybrać i ustawić trzy z nich?
	\item 
	%Wariacje z powtórzeniami
	%11. Układanie liter
	Na ile sposobów można utworzyć 3-literowe "słowa" z liter A, B, C, przy czym litery mogą się powtarzać?
	\item 
	%12. Kod PIN
	Tworzymy czterocyfrowy kod PIN, przy czym każda cyfra może być dowolnie powtórzona, a do wyboru mamy cyfry od 0 do 9. Na ile sposobów można stworzyć taki PIN?
	\item 
	%13. Kolory kulek
	Mamy do wyboru trzy kolory kulek: czerwony, zielony i niebieski. Na ile sposobów można ułożyć pięć kulek, jeśli mogą być one tego samego koloru?
	\item 
	%14. Liczby 2-cyfrowe
	Na ile sposobów można utworzyć liczbę 2-cyfrową, wybierając cyfry spośród 1, 2, 3, 4, przy czym cyfry mogą się powtarzać?
\end{enumerate} 

\subsection{Teoria grafów}
Do uzupełnienia.

\section{Uwagi lub (p)odpowiedzi}


\begin{itemize}
	\item Część~\ref{subsec:modulo}, zadanie~\ref{exc:modpowers}.
	Aby policzyć np. $3^{5^100} \bmod 7$ trzeba wykorzystać 
	twierdzenie Eulera, moulo równego $7$ i równego $6$.
	Zauważmy, że $\varphi(7)=6$, $\varphi(6)=2$.
	Po pierwsze, jeśli przedstawimy $5^{100} = 6k+i$, gdzie $0\leq i<6$, to 
	\[
	3^{6k+i} \equiv (3^6)^k3^i \equiv 1^k3^i \equiv 3^i \pmod 7.
	\]
	Teraz, aby sprawdzić, jaka jest reszta z dzielenia $5^100$ przez $6$, czyli aby znaleźć $i$ takie, że 
	\[
	5^{100}\equiv i \pmod 6.
	\]
	Ponieważ $5$ jest względnie pierwsze z $6$, z twierdzenia Eulera mamy $5^2 \equiv 1 \pmod 6$.
	Wtedy 
	\[
	5^100 \equiv (5^2)^{50} \equiv 1^{50} \equiv 1 \pmod 6.
	\]
	Skoro szukana wartość $i$ wynosi $1$, to 
	\[
	3^{5^100} \equiv 3^1 \equiv 3 \pmod 7.
	\]
    \item Część~\ref{ss:indukcja}, zadanie~\ref{ind:2nn2}. 
     
     W tezie $\forall n\geq 0 (2^n\geq n^2)$ nie uda się 
     udowodnić kroku indukcyjnego.
     
     W tezie $\forall n\geq 3 (2^n\geq n^2)$ krok indukcyjny da się udowodnić, ale nie da się udowodnić przypadek bazowy.
\end{itemize}

\bibliographystyle{alpha}
\bibliography{Mat_KGN}

\end{document}	
